\documentclass[landscape, twocolumn, 8pt, a4paper, twoside]{extarticle}

\usepackage{xeCJK, amsmath, amsthm, graphicx, amsfonts, fancyhdr, ctable, pict2e, CJKulem, multirow, rotating, geometry, listings}
\usepackage[CJKbookmarks, colorlinks = true, linkcolor = red, anchorcolor = blue, citecolor = blue, pdfstartview={FitH}]{hyperref}

\newcommand{\ud}{\mathrm{d}}

\renewcommand{\baselinestretch}{0.8}
\geometry{left=0.5cm, right=0.5cm, top=1.5cm, bottom=1.5cm}
\setCJKmainfont{STSong}
\lstset{language=C++, breaklines, frame=single,
    numbers=left,
    escapeinside=``,extendedchars=false,
    xleftmargin=2.0em,xrightmargin=2.0em,
    basicstyle=\ttfamily\small, tabsize=2,
}

\newcommand{\stlf}[2]{\genfrac{ [ }{ ] }{0pt}{}{#1}{#2}}
\newcommand{\stls}[2]{\genfrac{ \{ }{ \} }{0pt}{}{#1}{#2}}


\pagestyle{fancy}
\lhead{Shanghai Jiao Tong University}
\chead{\thepage}
\rhead{Tempest}
\lfoot{}
\cfoot{}
\rfoot{}

\title{Tempest - Standard Code Library}

\begin{document}

%\chapter{数论算法}
\section{$O(m^2\log n)$ 线性递推第n项}
	\input{main/linear_recurrence.tex}
\section{NTT}
	\input{main/NTT.tex}
\section{中国剩余定理}
	\input{main/chinese_remainder_theorem.tex}
\section{Miller Rabin}
	\input{main/miller_rabin.tex}
\section{Pollard Rho}
	\input{main/pollard_rho.tex}
\section{直线下整点个数}
	\input{main/直线下的整点个数.tex}

%\chapter{数值算法}
\section{FFT}
	\input{main/FFT.tex}
\section{解一元三次方程+求三阶二次型的标准型}
	\input{main/一元三次方程.tex}
\section{自适应辛普森}
	\input{main/adaptive_simpson.tex}

%\chapter{计算几何}
\section{圆与多边形交}
	\input{main/圆与多边形交.tex}
\section{动态凸包}
	\input{main/动态凸包.tex}
\section{farmland}
	\input{main/farmland.tex}
\section{半平面交}
	\input{main/half_plane_intersection.tex}
\section{三维绕轴旋转}
	\input{main/三维绕轴旋转.tex}
\section{$O(n\log k)$ 判断圆存在交集}
	\input{main/判断圆存在交集.tex}
\section{$O(n^2\log n)$ 圆交+计算面积和重心}
	\input{main/圆交.tex}
\section{三维凸包}
	\input{main/3D_convex_hull.tex}
\section{点在多边形内}
	\input{main/点在多边形内.tex}

%\chapter{数据结构}
\section{KD树}
	\input{main/KD树.tex}
\section{树链剖分}
	\input{main/树链剖分.tex}
\section{可持久化左偏树}
	\input{main/可持久化左偏树.tex}
\section{Treap}
	\input{main/treap.tex}
\section{坚固的Treap}
	\input{main/functional_treap.tex}
\section{LCT}
	\input{main/link_cut_tree.tex}
\section{Splay}
	\input{main/Splay.tex}

%\chapter{图论}
\section{Gabow算法求点双联通分量(非递归)}
	\input{main/Gabow.tex}
\section{$O(EV^{0.5})$ HK求二分图最大匹配}
	\input{main/Hopcroft_Karp.tex}
\section{$O(V^3)$ 最小树形图}
	\input{main/最小树形图.tex}
	%!!
\section{KM}
	\input{main/KM.tex}
%\section{度限制生成树}
\section{带花树}
	\input{main/一般图匹配.tex}
\section{无向图最小割}
	\input{main/无向图最小割.tex}
\section{哈密尔顿回路}
	\input{main/Hamilton回路.tex}
\section{弦图判定}
	\input{main/弦图判定.tex}
\section{弦图求团数}
	\input{main/弦图求团数.tex}
\section{ZKW费用流}
	\input{main/zkw费用流.tex}

%\chapter{字符串}
\section{扩展KMP}
	\input{main/扩展KMP.tex}
\section{后缀数组}
	\input{main/后缀数组.tex}
\section{DC3}
	\input{main/DC3.tex}
\section{AC自动机}
	\input{main/aho_corasick_automation.tex}
\section{极长回文子串}
	\input{main/极长回文子串.tex}
\section{后缀自动机 多次询问串在母串中出现的次数}
	\input{main/后缀自动机--多次询问串在母串中的出现次数.tex}
\section{循环串的最小表示}
	\input{main/SCL_str_cyc_min_rep.tex}

%\chapter{其他}
\section{快速求逆}
	\input{main/快速求逆.tex}
\section{求某年某月某日是星期几}
	\input{main/求某年某月某日星期几.tex}
\section{LL*LL\%LL}
	\input{main/LLMOD.tex}
\section{next\_nCk}
	\input{main/next_nCk.tex}
\section{单纯形}
	\input{main/单纯形.tex}
\section{环状最长公共子序列}
	\input{main/环状最长公共子序列.tex}
\section{长方体表面两点最近距离}
	\input{main/长方体表面最近距离.tex}
\section{插头DP}
	\input{main/插头DP.tex}
\section{极大团搜索}
	\input{main/最大团搜索.tex}
\section{Dancing Links X}
	\input{main/Dancing_Links.tex}

%\chapter{提示}
\section{积分表}
	\input{main/积分表.tex}
\section{网络流}
	下界:(u, v)下界为c:超级源到t建流量为c, s到超级汇建流量为c, (原来的汇到原来的源建无穷, 如果有), 流一遍超级源出边满了就存在可行流.\\
	下界最大流(有源汇): 上面的搞完从原来的源到原来的汇流一遍\\
	下界最小流(有源汇): 上面的搞完从原来的汇到原来的源流一遍\\
\section{2-SAT}
	每对点都选择强连通时color较小的\\
\section{二分图}
	\input{main/二分图.tex}
\section{Java}
	\input{main/java_template.tex}
\section{Rope}
	\input{main/Usage_of_Rope.tex}

\end{document}